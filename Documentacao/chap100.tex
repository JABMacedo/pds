
\chapter{Introdução}


\section{Contexto}



\section{Objetivos}


\begin{itemize}
	\item inserir obejtivos

\end{itemize}

\clearpage

\section{Estrutura do documento}
Este documento agrupa toda a documentação produzida pelo grupo de trabalho em todas as fases do projeto.

\begin{enumerate}
	\item Produto. Contextualizar o problema e fazer o enquadramento do mesmo;

\end{enumerate}

\subsection*{\textbf{NOTA}:}

Cada um dos capítulos da lista anterior inclui ainda dados referentes a todas as reuniões do grupo de trabalho. A parte alfanumérica que identifica cada reunião corresponde a cada \textit{milestone} do projeto:
\begin{itemize}
	\item \textbf{ES} = Especificação;
	\item \textbf{MA} = Release Alpha;
	\item \textbf{MB} = Release Beta;
	\item \textbf{RTW} = Release to Web. . 
\end{itemize}

\newpage

\section{Equipa}

A composição do grupo de trabalho, nome do projeto, nome da equipa e cargos foi finalizada na primeira reunião (identificada como reunião MA01 transcrita no capítulo \ref{reuniaoMA01} na página \pageref{reuniaoMA01}).\\[4mm]

\noindent \textbf{Nome do projeto}\\
\noindent \rule{\linewidth}{0.4pt}
\noindent Introduzir nome do projeto \\

\noindent \textbf{Nome da equipa}\\
\noindent \rule{\linewidth}{0.4pt}
\noindent nome da equipa/empresa \\

\noindent \textbf{Membros e cargos}\\
\noindent \rule{\linewidth}{0.4pt}
\noindent nome (numero) \textbf{Product Owner}\\
\noindent nome (numero) \textbf{Scrum Master}\\
\noindent nome (numero) Development team member\\
\noindent nome (numero) Development team member\\
\noindent nome (numero) Development team member\\
