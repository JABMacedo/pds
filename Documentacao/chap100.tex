
\chapter{Introdução}

A necessidade de automatizar e simplificar acompanham a necessidade de orzanizar os documentos e da sua rastreabilidade. A informatização dos processos administrativos são cada vez mais importantes nos dias de hoje de maneira a deixar mais tempo livre os intervenientes para outras tarefas que não sejam possiveis de informatizar. 

\section{Contexto}

Este projeto nasce no âmbito curricular, interligando as \acrshort{ucs} de \acrshort{ucpds} e \acrshort{ucpw}. Desta maneira nasce o projeto \textit{I-Library} que consiste em criar uma plataforma direcionada à biblioteca do IPCA de gestão e requisição de livros. 


\section{Objetivos}

O objetivo deste projeto é simular uma biblioteca online, onde é possivel fazer requisições de livros online, fazer pesquisa dos livros existentes na biblioteca do IPCA.

Facilitar ao aluno um espaço online, onde não é preciso a presença do mesmo na biblioteca, e de manter um progresso e histórico dos livros que o mesmo requisitou, pesquisou ou até mesmo se esqueceu de entregar.

Facilitar ao funcioário da biblioteca um espaço online, onde ele pode fazer um controlo do stock online e uniforme. Sendo online pode também dar baixa de livros e gerir com mais rapidez e eficácia todas as requisições ativas e coimas ativas da biblioteca.

Facilitar ao admin/gestor da biblioteca um espaço online, onde tem todo o controlo sobre a biblioteca online, sem precisar de estar sempre presencialmente. Pode gerir os funcionários/alunos e todos os livros, requisições e coimas de toda a biblioteca com poucos cliques.

\clearpage

\section{Estrutura do documento}
Este documento agrupa toda a documentação produzida pelo grupo de trabalho em todas as fases do projeto.

\begin{enumerate}
	\item Especificação. Contextualizar o problema e fazer o enquadramento do mesmo;
	\indent Planear as sprints inici, análise de requisitos Planeamento
	inicial de sprints, especificação das regras de negócio,
	desenvolvimento dos wireframes.
	\item Release Alpha 
	\item Release Beta
	\item Release to Web

\end{enumerate}

\subsection*{\textbf{NOTA}:}

Cada um dos capítulos da lista anterior inclui ainda dados referentes a todas as reuniões do grupo de trabalho. A parte alfanumérica que identifica cada reunião corresponde a cada \textit{milestone} do projeto:
\begin{itemize}
	\item \textbf{ES} = Especificação;
	\item \textbf{MA} = Release Alpha;
	\item \textbf{MB} = Release Beta;
	\item \textbf{RTW} = Release to Web. . 
\end{itemize}

\newpage

\section{Equipa}

A composição do grupo de trabalho, nome do projeto, nome da equipa e cargos foi finalizada na primeira reunião.

\noindent \textbf{Nome do projeto}\\
\noindent \rule{\linewidth}{0.4pt}
\noindent I-Library \\

\noindent \textbf{Nome da equipa}\\
\noindent \rule{\linewidth}{0.4pt}
\noindent 5Solutions \\

\noindent \textbf{Membros e cargos}\\
\noindent \rule{\linewidth}{0.4pt}
\noindent Vitor Sá (20484) \textbf{Product Owner}\\
\noindent Jonatas de Paula  (22562) \textbf{Scrum Master}\\
\noindent José Macedo (12939) Development team member\\
\noindent João Machado (21151) Development team member\\
\noindent Daniel Vilaça (16939) Development team member\\
