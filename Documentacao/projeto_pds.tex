\documentclass[a4paper,12pt,twoside]{book}
\usepackage{lesi/lesi}


\title{Projeto de Desenvolvimento de Software}
\author{José Macedo, 12939 \AND Daniel Vilaça, 16939 \AND  Vitor Sá, 20484 \AND João Machado, 21151 \AND Jonatas de Paula, 22562}
\LESI
\regimePosLaboral
\date{\today}

\orientador{Nuno Rodrigues}

% comentar se não for para usar glossários
%\makeglossaries


% inicio do documento
\begin{document}
% 
\newglossaryentry{exemplo}
{
	name=exemplo,
	description={Exemplo de descrição}
}
 

\newacronym{pds}{PDS}{Projeto de Desenvolvimento de Software}

\newacronym{pw}{PW}{Programação Web}

\newacronym{ucisi}{ISI}{Integração de Sistemas Informáticos}

\newacronym{uc}{UC}{Unidade Curricular}

\newacronym{ipca}{IPCA}{Instituto Politécnico do Cávado e do Ave}



\frontmatter
\maketitle  % print the title

\begin{resumo}
Fazer um pequeno resumo sobre o projeto...
\end{resumo}

\tableofcontents

% comentar se nao tiver figuras
\listoffigures
\addcontentsline{toc}{chapter}{Lista de Figuras}

% Commentar proximas duas linhas se nao for para usar acronimos
\printglossary[type=\acronymtype,title={Siglas \& Acrónimos},toctitle={Siglas \& Acrónimos}]

% Commentar proximas duas linhas se nao for para usar glossarios
\printglossary[title={Indice de Termos},toctitle={Indice de Termos}]

% comentar se nao se quiser lista de listagens
%\lstlistoflistings
%\addcontentsline{toc}{chapter}{Lista de Código}

% comentar se não for para usar bibliografia
%\bibliography{biblio}
%\addcontentsline{toc}{chapter}{Bibliografia}

\mainmatter

% introdução

\chapter{Introdução}


\section{Contexto}



\section{Objetivos}


\begin{itemize}
	\item inserir obejtivos

\end{itemize}

\clearpage

\section{Estrutura do documento}
Este documento agrupa toda a documentação produzida pelo grupo de trabalho em todas as fases do projeto.

\begin{enumerate}
	\item Produto. Contextualizar o problema e fazer o enquadramento do mesmo;

\end{enumerate}

\subsection*{\textbf{NOTA}:}

Cada um dos capítulos da lista anterior inclui ainda dados referentes a todas as reuniões do grupo de trabalho. A parte alfanumérica que identifica cada reunião corresponde a cada \textit{milestone} do projeto:
\begin{itemize}
	\item \textbf{ES} = Especificação;
	\item \textbf{MA} = Release Alpha;
	\item \textbf{MB} = Release Beta;
	\item \textbf{RTW} = Release to Web. . 
\end{itemize}

\newpage

\section{Equipa}

A composição do grupo de trabalho, nome do projeto, nome da equipa e cargos foi finalizada na primeira reunião (identificada como reunião MA01 transcrita no capítulo \ref{reuniaoMA01} na página \pageref{reuniaoMA01}).\\[4mm]

\noindent \textbf{Nome do projeto}\\
\noindent \rule{\linewidth}{0.4pt}
\noindent Introduzir nome do projeto \\

\noindent \textbf{Nome da equipa}\\
\noindent \rule{\linewidth}{0.4pt}
\noindent nome da equipa/empresa \\

\noindent \textbf{Membros e cargos}\\
\noindent \rule{\linewidth}{0.4pt}
\noindent nome (numero) \textbf{Product Owner}\\
\noindent nome (numero) \textbf{Scrum Master}\\
\noindent nome (numero) Development team member\\
\noindent nome (numero) Development team member\\
\noindent nome (numero) Development team member\\


% Release Especificação
\chapter{Especificação}

Neste capítulo abordaremos os aspetos que o nosso produto possui.



% Release Alpha

\chapter{Release Alpha}

%planeamento & sprints
\input{chap301}

%reuniões
\input{chap350}

% Release Beta

\chapter{Release Beta}

%planeamento & sprints
\input{chap401}

%reuniões
\input{chap450}

% Release ready to web

\chapter{Release to Web}

%planeamento & sprints
\input{chap501}

%reuniões
\input{chap550}


\end{document} 